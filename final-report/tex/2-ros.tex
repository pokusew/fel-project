\chap[ros] Robot Operating System

The Robot Operating System (or ROS) is “an open-source, meta-operating system”~\cite[ros1_intro]
that consists of “tools, libraries, and conventions that aim to simplify the task of creating complex and
robust robot applications across a~wide variety of robotic platforms“~\cite[ros2_foxy_root].

a~typical ROS application is composed of loosely coupled processes – {\sbf nodes} – (potentially distributed across
machines) that communicate with each other and work together to accomplish a~certain goal (for example,
autonomous vehicle control). Each node performs only a~limited set of specific functions {\em(e.g.: one node
can read data~from LIDAR, another node can implement an obstacle detection algorithm from the LIDAR data,
while another node can use the detected obstacles to plan the trajectory, etc.)}.

Such architecture greatly supports the separations of concerns and allows code reuse. That further enables
efficient code sharing of common functionality among different projects with various applications. That in
fact, is one of the most significant features of ROS as there are thousands of ROS packages provided by the
ROS community~\cite[ros_index_stats]. Developers can focus on their application-specific problems while reusing code
for common parts.


\sec[ros_graph] ROS Computation Graph

At runtime, ROS nodes and their communication interactions form so called “ROS Computation
Graph”~\cite[ros1_concepts]. The nodes are represented by the graph's vertices, while the edges
depicts the communication interactions.

\midinsert
\clabel[img_ros_graph]{a~simple ROS Computation Graph}
\picw=12cm \cinspic ../images/ros-simple-graph.pdf
% Note: Footnotes do NOT work in boxes (tables, captions).
%       See Section 1.2.3 in OpTeX Manual where workaround is described.
\caption/f a~simple ROS Computation Graph. Two nodes (represented as ellipses) ("/turtlesim" and "/teleop_turtle")
from are running. The rectangles represent topics. The application is a~part of the "turtlesim"\fnotemark1 package.
\endinsert

\fnotetext{\url{https://index.ros.org/p/turtlesim/}}


\secc[ros_communication] ROS Communication Primitives

ROS provides several {\em communication primitives} that can be used by nodes:
\begitems \style n

* {\sbf messages and topics} – publish/subscribe \nl
Nodes (publishers) can publish messages to a~named topic. Other nodes (subscribers) can subscribe to a
topic and receive the published messages.

* {\sbf services} – synchronous RPC (Remote Procedure Call) (server/client) \nl
Nodes (service servers) can provide services. a~service has a~name. Other nodes (service clients) can
invoke/call a~service and synchronously get a~result.

* {\sbf actions} – asynchronous preemptible RPC with continuous feedback (server/client) \nl
Node (action servers) can provide actions. An action is preemptible task that has a~goal, can provide
continuous feedback and returns a~result if it is not cancelled. Other nodes (action clients) can invoke an
action (request a~goal and subscribe for its feedback and result).

\enditems

The communication is strongly typed and ROS provides an interface description language (IDL) for describing
message types (used for pub/sub), service types and action types.


\sec[ros_concepts] ROS Common Concepts

In addition to the Computation Graph and communication possibilities, both versions of ROS share many additional common
concepts, some of which are described below:
\begitems

* {\sbf package} – a~container for code, IDL files (messages, services, actions), configuration files or anything
else. It is “the most atomic build and release item” that groups together common functionality that can be
easily shared and reused~\cite[ros1_concepts]. Each package has a~package manifest file "package.xml"
that provides “metadata~about a~package, including its name, version, description, license information,
dependencies, etc.“. There are currently 3 package manifest formats, which are defined
in REP-127\urlnote{https://ros.org/reps/rep-0127.html}, REP-140\urlnote{https://ros.org/reps/rep-0140.html},
REP-149\urlnote{https://ros.org/reps/rep-0149.html} respectively.

* {\sbf distribution} – “a~versioned set of ROS packages~\cite[ros1_distros]. ROS~1 distributions are
listed at~\cite[ros1_distros], ROS~2 distributions are listed at~\cite[ros2_distros].

* {\sbf workspace} – a~directory containing packages that are built together using a~ROS build tool
(such as "catkin_make", "catkin_make", "catkin_make_isolated", "catkin_tools", "colcon") and a~build system
(such as "catkin" or "ament").

* {\sbf Graph Resource Names} – “a~hierarchical naming structure that is used for all resources in a~ROS
Computation Graph, such as Nodes, Parameters, Topics, and Services“~\cite[ros1_concepts].

* {\sbf Package Resource Names} – a~simplified method of “referring to files and data~types on
disk“~\cite[ros1_concepts]. It consists of the name of the package that the resource is in plus
the name of the resource. For example, the name "std_msgs/String" refers to the "String" message type in the
"std_msgs" package.

* {\sbf ROS client library} – a~collection of code that simplifies the task of implementing of a~ROS node in a
certain programming language. “It takes many of the ROS concepts and makes them accessible via
code”~\cite[ros1_client_libs]. It provides an API (functions, methods, classes) that allows
to interact with other nodes (using pub/sub, services, actions and parameters) and do other common things.
The main ROS client libraries are for C++ and for Python. The internal architecture of client libraries differs
greatly between ROS~1 and ROS~2.

\enditems


\sec ROS~1

ROS~1 is the original version of ROS that dates back to 2007. The version 1.0 was published in 2010. Since
then, 13 ROS~1 distributions have been released\fnote{As of 28th May 2020}. The full list can be found
at~\cite[ros1_distros].

Full documentation of ROS~1 can be found at ~\cite[ros1_docs].


\secc Architecture

In ROS~1 ROS Computation Graph, there must always be a~special node called {\em ROS~Master}. It provides name
registration and lookup services to the rest of the Computation Graph. It coordinates the communication among
the nodes. That includes graph changes notifications and establishing of connections connection between nodes.
It offers two XML-RPC based APIs – Master API and Parameter Server API.

ROS~1 has a~concept of central data~key/value data~storage called {\em Parameter~Server}. It is currently part of
ROS Master (Parameter Server API). a~key/value pair is called parameter. All nodes in the Computation Graph
can get and manipulate the key/value data~stored in the Parameter Server. The parameters can be used as
configuration storage for nodes, which allows easy altering of the system (nodes') behavior at runtime.

ROS~1 offers two underlying data~transports protocols – {\em TCPROS} and {\em UDPROS}. As the names suggest,
they are based on TCP and UDP respectively.

	{\sbf All concepts are implemented directly (natively) in ROS~1 Client Libraries.} The two main client libraries
are {\sbf roscpp} (for C++) and {\sbf rospy} (for Python). Their performance and features availability varies
greatly. While the C++ ROS~1 Client Library implements all ROS~1 features and provides high performance, the
Python ROS~1 Client Library lacks some features and provides worse performance. Even when a~feature is
available in both libraries, the actual implementation often comes with minor differences that might not be
expected.

% TODO: It would be nice to have ROS~1 architecture digram here (user code + client lib)


\secc Build System

ROS~1 uses catkin build system\fnote{In older ROS~1 distributions, rosbuild was used. It is also (theoretically)
	possible to use ROS~2 build tool colcon to build ROS~1 workspace.}. Catkin supports CMake packages that uses
special catkin CMake macros. The actual build is controlled by a~build tool. ROS~1 catkin
workspace can be built using different build tools – "catkin_make", "catkin_make_isolated", "catkin_tools".


\secc CLI

The CLI is composed of several commands that covers all ROS~1 features. These include for example
"rostopic", "rosservice", "rosparam", "rosmsg", "rosrun", "roslaunch", etc. ~\cite[ros1_cli].


\secc Launch System

While nodes can be started manually (via~running the corresponding executable), for complex system it may be
cumbersome. For this reason, ROS~1 offers allows to describe a~system using a~special XML file. Then the
command "roslaunch" handles starting up all nodes and supplying correct arguments~\cite[ros1_launch].


\sec ROS~2

“Since ROS was started in 2007, a~lot has changed in the robotics and ROS community. The goal of the ROS~2
project is to adapt to these changes, leveraging what is great about ROS~1 and improving what is
not.”~\cite[ros2_foxy_root]

Full documentation of the latest ROS~2 release can be found at ~\cite[ros2_docs].


\secc Architecture

The architecture of ROS~2 was designed from the ground up addressing issues of ROS~1~\cite[ros2_design_why].
The newly designed architecture should address new use-cases as well as many issues from ROS~1:
\begitems
* truly distributed system (no master node)
* support for real-time
* more nodes in one process (composable nodes)
* better support for communication in non-ideal networks
* small embedded platforms support
\enditems

\midinsert
\clabel[img_ros2_architecture]{ROS 2 Architecture}
\picw=14.5cm \cinspic ../images/ros2-client-library-api-stack.png
\caption/f ROS 2 Architecture~\cite[ros2_foxy_internal]
\endinsert

While ROS~1 communications stack is built almost entirely from scratch, ROS~2 relies on {\em Data~Distribution
Service} (DDS). DDS is “a~middleware protocol and API standard for data-centric connectivity from the Object
Management Group (OMG). It provides low-latency data~connectivity, extreme reliability, and a~scalable
architecture for Internet of Things applications need”~\cite[omg_what_is_dds].

ROS~2 Client Libraries have a~different architecture compared to the ROS~1 ones. Instead of reimplementing all
the features in programming language separately, the common functionality is implemented in {\em rcl} library that
exposes a~C API. Client Libraries then use {\em rcl} library and implement the rest of the features on top of it
(in particular, language-dependent features, such as threading and execution model).

Because ROS~2 supports different DDS implementations from different vendors~\cite[ros2_foxy_dds],
rcl library cannot directly communicate with the DDS implementation. Instead, there is an abstraction layer
called  {\em rmw} (ROS middleware) that provides a~unified accesses to DDS. The specific DDS implementation can be
even supplied dynamically at runtime.

The whole relationship among different parts of ROS~2 Client Libraries
is shown in the Figure~\ref[img_ros2_architecture].
Additional detailed information can be found at~\cite[ros2_foxy_internal].


\secc Build System

ROS~2 uses ament as build system and colcon as build tool. Ament supports three types of packages: CMake with
ament_cmake, pure Python packages (Python setuptools based) and pure CMake packages. Colcon always builds all
packages in isolation and in a~correct order. For ensuring the correct build order, a~dependencies graph is
constructed which must be a~directed acyclic graph in order for the workspace to be buildable. It also tries
to parallelize the build up to the level that mutual dependencies among packages allow.


\secc CLI

It is very similar to ROS~1 CLI but instead of having multiple commands, ROS~2 has one central command "ros2"
that has several subcommands.

\secc Launch System

ROS 2 Launch System was redesigned. Currently, it uses Python 3 code to describe the system to
launch~\cite[ros2_foxy_launch, ros2_design_launch].

\medskip

\sec Summary of Differences between ROS~1 and ROS~2

ROS~2 has better architecture and offers more features. Thanks to DDS and its fine-grained QoS, ROS 2 handles
communication in non-ideal networks better than ROS 1. Furthermore, the ROS~2 client libraries offer better control
over code execution and threading as they support writing custom executors. ROS~2 architecture is designed with
real-time support in mind\fnote{Although the actual Real-Time properties may differ dramatically based on various
configuration and threading/execution model choice.}. The following list summarizes the most notable differences
between ROS~1 and ROS~2:

% TODO: Tables do not support multiline text in cells?

% TODO: Improve the differences list.

\begitems

* Supported Platforms
\begitems \style x
* {\em ROS~1:} Only Ubuntu officially supported.
* {\em ROS~2:} Ubuntu, macOS, Windows officially supported. Yet, a lot of packages work only on Ubuntu.
\enditems

* Client Libraries
\begitems \style x
* {\em ROS~1:} roscpp (C++03), rospy (Python 2, in later distributions Python 3)
* {\em ROS~2:} rclcpp (C++14), rclpy (Python 3)
\enditems

* Real-Time Support
\begitems \style x
* {\em ROS~1:} no
* {\em ROS~2:} yes
\enditems

* Runtime Node Composition
\begitems \style x
* {\em ROS~1:} No. But Nodelets can be used.
* {\em ROS~2:} Yes. Composable Nodes.
\enditems

* Parameters
\begitems \style x
* {\em ROS~1:} global parameter server
* {\em ROS~2:} parameters per node (no global parameter server), out-of-the-box dynamic_reconfigure-like features
\enditems

* Launch System
\begitems \style x
* {\em ROS~1:} XML-based
* {\em ROS~2:} Python-based
\enditems

* Transport
\begitems \style x
* {\em ROS~1:} TCPROS or UDPROS
* {\em ROS~2:} Handled by DDS which offers fine-grained QoS.
\enditems

* Communication Primitives
\begitems \style x
* {\em ROS~1:} pub/sub, services, actions (not natively, but via actionlib)
* {\em ROS~2:} pub/sub, services, actions
\enditems

* Threading and Execution Model
\begitems \style x
* {\em ROS~1:} not much customizable
* {\em ROS~2:} granular execution models, custom executors
\enditems

* Build
\begitems \style x
* {\em ROS~1:} catkin + catkin_make/catkin_make_isolated
* {\em ROS~2:} ament + colcon
\enditems

* IDL
\begitems \style x
* {\em ROS~1:} ".msg"/".srv"
* {\em ROS~2:} ".msg"/".srv"/".action" + extended features such as constraints
\enditems

\enditems
