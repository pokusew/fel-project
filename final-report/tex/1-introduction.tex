\chap[intro] Introduction

Since the Robot Operating System was started in 2007~\cite[ros_history], it has gained great popularity and
has become the standard in robotics community. However, it has turned out that the original architecture
(ROS~1) has some limitations concerning {\em performance, efficiency and real-time safety}. Thus, a~new
version of ROS -- {\em ROS~2} -- has been designed from ground up to allow more use-cases and solve pain points of
ROS~1 ~\cite[ros2_design_why]. Missing features and incompatible packages have been slowing down the adoption of
ROS~2 since its first public release in 2017. But in recent years, the situation has improved a~lot, and the
adoption of ROS~2 has accelerated~\cite[ros_metrics]. Thus, now it might be the right time start
migrating applications from ROS~1 to ROS~2 and benefit from new possibilities.

The F1/10 platform is scaled-down (1:10) model of an autonomous car that originates from F1/10 Autonomous
Racing Competition. Thanks to its affordability, the platform can be used for development, testing and
verification of autonomous driving systems and related algorithms. At CTU, multiple F1/10 models have been
created and used ~\cite[vajnar_f1tenth_2017, dusil_detection_2019, klapalek_avoidance_2019]. {\em Currently},
all of them are based on NVIDIA Jetson computing modules and powered by ROS~1 Kinetic Kame.

{\sbf The goal} of this project is to migrate a~selected part of the CTU's F1/10 project from ROS~1 Kinetic Kame
to ROS~2 Foxy Fitzroy. The result should be a~working port running on ROS~2 in a~simulator and on a~physical
model car with NVIDIA Jetson computing module.

In the Chapter~\ref[ros], both versions of ROS are briefly described while highlighting main differences of ROS~2
when compared to ROS~1.

In the Chapter~\ref[f1tenth], the software and hardware stack of CTU's F1/10 project is introduced. Then the Follow the
Gap application is selected for migration to ROS~2. The migration is described. The working of the migrated
application is demonstrated in the Stage simulator on Ubuntu and macOS.

The Chapter~\ref[jetson] focuses on setting up an NVIDIA Jetson TX2 module so that it can run ROS~2 applications. The
emphasis is given on providing a~good developer experience in environments where the Jetson module is shared
among many students and researches. Thus, boot options of the Jetson modules are explored. Then a~solution
with a~bootable SD card is presented.

Finally, {\sbf the achieved results} are summarized in the last Chapter~\ref[conclusion].
