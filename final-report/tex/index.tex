\input ctustyle3

\worktype [O/EN]
\workname {Semestral Project}

\faculty {F3}
\department {Department of Cybernetics}

\title {Migration of F1/10 Autonomous Driving Stack to ROS 2}
\author {Martin Endler}
\date {May 2021}

\supervisor {Ing. Michal Sojka, Ph.D.}
\studyinfo  {Open Informatics – Artificial Intelligence and Computer Science}
\workinfo   {\url{https://github.com/pokusew/fel-project}}

\abstractEN {

	The new version of the Robot Operating System – ROS 2 – brings significant improvements
	over its predecessor \hbox{ROS 1}. We briefly describe both versions and compare their differences.
	Then we cover the process of migrating a ROS 1 application to ROS 2 on the example of the Follow the Gap app
	that is part of CTU's F1/10 project.

	In the next part, we focus on setting up an NVIDIA Jetson TX2 module so that it can run ROS 2 applications and
	provide good developer experience in environments where Jetson module is shared among many students and
	researches.

	The first result of this project is \hbox{a ROS 2} port of the Follow the Gap application. Its working is
	demonstrated in the Stage simulator on Ubuntu and macOS. The second result is a setup guide for NVIDIA Jetson
	TX2 that covers creating a fully-setup OS image with ROS 2 on a bootable SD card. The last result is a
	collection of setup guides and documentation that cover various aspects of working with ROS.

}
\keywordsEN {
	ROS, ROS 1, ROS 2, ROS 2 migration, F1/10, Follow the Gap, autonomous model car, NVIDIA Jetson TX2
}

\thanks {
	I would like to thank everybody. % TODO: Acknowledgement
}
\declaration {
	I declare that the presented work was developed independently and that I have listed all sources of information
	used within it in accordance with the methodical instructions for observing the ethical principles in the
	preparation of university theses.
}

\makefront

\chap Introduction
TODO

\chap ROS
\sec ROS 1
TODO

\chap Conclusion
TODO

\bye
